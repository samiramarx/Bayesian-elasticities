\chapter{Introduction}
\label{chp:introduction}

% PROBLEM - GENERAL
According to economic theory, for a normal good, the own price-elasticity of demand is expected to be a negative value, and the cross-elasticities of its competitive goods are expected to assume positive values. The estimation of these values with the aid of regression models using sales data and the conventional sampling theory approach, however, often produces frustrating wrong algebraic signs outputs \citep{liu2009}. The problem of estimation of own and cross price elasticities of demand is already acknowledged in the literature (\cite{griffiths1988}, \cite{geweke1986}, \cite{montgomery1999}, \cite{liu2009}).

% PROBLEM - RAIL INDUSTRY
In the rail industry, this problem emerges in the estimation of fare elasticities of demand. The freedom to commercially explore fares, establishing market-based prices, and the traditional practice of price discrimination in the rail industry enabled the existence of a wide range of different tickets types. However, identifying how these tickets compete with one another and how their price change affects theirs each other demand has been a challenging task. As reported in previous studies, some works based on ticket sales data have ``failed to specify cross elasticities whilst those that did often found them to be either wrong sign or statistically insignificant" \citep[p.~6]{wardman2003}.

The main root of this problem is being considered as the ``high degree of correlation between the fares of different tickets" \citep[p.~6]{wardman2003}, due to the practice of annual fare revisions, which ``compounds the already difficult task of estimating what are relatively small effects" \citep[p.~6]{wardman2003}.

Different approaches have been tried to overcome the data correlation issue. Theoretical constraints and estimation procedures mixing revealed preference and stated preference data can be named as the main alternatives (\cite{wardman2003} and \cite{its-systra-report}). Nevertheless, there still is a lack of a methodology to be widely satisfactory applicable across different markets to bring coherent and consistent elasticity estimates. 

% OBJECTIVES
Because of those difficulties, fare elasticities are considered to ``probably represent the most important area of disagreement on rail demand forecasting" \citep{its-systra-met}, embodying a relevant problem of research. This study aims to expand the spectrum of approaches to estimate fare elasticities in the Great Britain introducing Bayesian inference.

% BAYESIAN ECONOMETRICS
Bayesian econometrics bring considerable advantages. In addition to providing ``a more natural interpretation of the results of a statistical investigation than does the sampling theory", it offers a ``formal framework for incorporating prior information" \citep[p.~36]{griffiths1988} available from economic theory. Another appealing feature regards the straightforward method to apply inequality constraints, contrasting to the quadratic programming alternative \citep{geweke1986}. 

It worths highlighting that Bayesian methods have been a useful tool in marketing, which can be noticed from the discussion in the literature, and commercially, with practical application in retail price optimization \citep{liu2009}. Leading American retailers - as Target, Walmart, Safeway and Giant Eagle - already make use of Bayesian methods to optimize profits by product category \citep{liu2009}.

Therefore, the application of Bayesian methods to estimate rail fare elasticities may contribute to the body of knowledge that has been built so far and also contributes to the improvement of the rail industry. Even though this work does not reach promising techniques of Bayesian econometrics - as hierarchical models, it will open the debate performing Bayesian regression models with the aim of achieving coherent algebraic sign estimates - the most elementary issue of estimating estimates.

% STRUCTURE
This work will be structured in five chapters. Following this introduction, chapter two covers the literature review in which the problem of estimation of price effects is debated more deeply, both the general aspects and the specific issues in the rail industry. Additionally, previous studies are revisited to present a holistic view of the evidence so far. Still, a section is dedicated to exploring the Bayesian rationale, which is the basis of this study, and how its application is related to the research problem. 

Chapter three comprises the methods applied in this work. It presents information regarding the dataset and the definition of the econometric model. For being an inherent characteristic of the Bayesian inference, sections are dedicated to justifying the choice of prior densities of the model's parameters and the likelihood function. 

Chapter four regards the results. It will be assessed the consistency of the estimates' signs and compared with the ordinary least square estimates. Additionally, the uncertainty of the estimation is also appraised, discussed in terms of the range of credible values they may assume - the \textit{high density intervals}. Lastly, an initial debate about the magnitude of estimates is attempted. It will be more descriptive and superficial since a proper analysis of the 95 estimated coefficients holds a complexity that can not be comprised in this work.

In the last chapter, conclusions are discussed and further developments are suggested. 
